\documentclass[14pt]{extarticle}

\usepackage{cmap}
\usepackage[T2A]{fontenc}
\usepackage[utf8]{inputenc}
\usepackage[russian]{babel}
\usepackage{graphicx}
\usepackage{amsthm,amsmath,amssymb}
\usepackage[russian,colorlinks=true,urlcolor=red,linkcolor=blue]{hyperref}
\usepackage{enumerate}
\usepackage{datetime}
\usepackage{minted}
\usepackage{fancyhdr}
\usepackage{lastpage}
\usepackage{color}
\usepackage{verbatim}
\usepackage{amsthm,amsfonts,amscd}
\usepackage{blkarray}
\usepackage{mathtools}
\usepackage{xcolor,cancel}
\usepackage{tikz}
\usepackage{ifthen}
\usepackage{setspace}
\usepackage{epigraph}
\usepackage{enumitem}

\parskip=0em
\parindent=0em

\sloppy
\voffset=-35mm
\textheight=235mm
\hoffset=-15mm
\textwidth=180mm
\headsep=12pt
\footskip=20pt

\begin{document}
Программа --- последовательность инструкций

$PROGRAM \to SEQ$

$SEQ \to (INSTRUCTION\ WS^+)^*$

$INSTRUCTION \to IF | ASSIGN | WHILE | READ | WRITE$

Есть 5 типов инструкций:
\begin{enumerate}
    \item Условный переход. Записывается с помощью ключевого слова \textit{if}, отделенное
          произвольным количеством пробелов, за которым в круглых
          скобках следует условие в виде выражения. Далее следует блок кода, выполняющийся если условие
          в скобках истинно. Перед блоком также может быть произвольное количество пробелов.

          Блок кода --- последовательность инструкций, заключенная в фигурные скобки, после
          открывающей и перед закрывающей может быть произвольное количество пробельных символов.

          $BLOCK \to \{WS ^ *\ SEQ\ WS ^*\}$

          Также после этой закрывающей скобки может быть ключевое слово \textit{else}, отделенное
          произвольным количеством пробелов, после которого через произвольное количество пробелов
          следует блок кода, который выполняется если условие ложно.

          $IF \to \textit{if}\ SPACE^*\ LBR\ EXPR\ RBR \ SPACE^*\ BLOCK$

          $(WS^+\ \textit{else }\ WS ^* \ BLOCK)?$

    \item Цикл \textit{while}. Записывается с помощью ключевого слова \textit{while}, отделенное
          произвольным количеством пробелов,за которым в круглых
          скобках следует условие в виде выражения. Далее следует блок кода, выполняющийся пока условие
          в скобках истинно. Перед блоком также может быть произвольное количество пробелов.

          $WHILE \to \textit{while } SPACE ^ *\ LBR\ EXPR\ RBR \ SPACE ^ *\ BLOCK$

    \item Присваивание. Записывается как идентификатор, затем произвольное количество пробелов, знак
          равенства, затем выражение, затем точка с запятой. Описание идентификаторов и выражения ниже.

          $ASSIGN \to IDENT\ SPACE^*\ \textbf{=}\ SPACE^*\ EXPR\text{;}$

    \item Чтение переменной. Записывается как \textit{read}, затем в скобках пишется идентификатор, в
          конце точка с запятой.

          $READ \to \textit{read } LBR\ IDENT \ RBR \text{;}$

    \item Вывод значения выражения. Записывается как \textit{write}, затем в скобках выражение, в
          конце точка с запятой.

          $WRITE \to \textit{write } LBR\ EXPR \ RBR \text{;}$
\end{enumerate}
Идентификатор начинается в буквы латинского алфавита или подчеркивания. далее следует произвольное количество
букв. цифр и знаков нижнего подчеркивания.

$IDENT \to ((a..z) | (A..Z) | \_ ) ((a..z) | (A..Z) | \_ | (0..9)) ^ *$

Выражения конструируются из чисел, переменных (идентификаторов), скобок и операторов естественным образом.

\textbf{Пробельные символы в выражениях на данный момент не поддерживаются!}

Приоритеты и ассоциативность операторов перечислены в таблице
\\

\begin{tabular}{|c|c|c|c|}
    \hline
    \textbf{Оператор} & \textbf{Тип} & \textbf{Приоритет} & \textbf{Ассоциативность} \\
    \hline
    \textasciicircum  & Бинарный     & Высший             & Правоассоциативен        \\
    \hline
    $-$               & Унарный      &                    &                          \\
    \hline
    $*,/$             & Бинарный     &                    & Левоассоциативен         \\
    \hline
    $+,-$             & Бинарный     &                    & Левоассоциативен         \\
    \hline
    $==,!=,<,>,<=,>=$ & Бинарный     &                    & Не ассоциативен          \\
    \hline
    $!$               & Унарный      &                    &                          \\
    \hline
    $\&\&$            & Бинарный     &                    & Правоассоциативен        \\
    \hline
    $||$              & Бинарный     & Низший             & Правоассоциативен        \\
    \hline
\end{tabular}
\\

Формальная грамматика выражений:
\begin{gather*}
    EXPR \to OR\_E\\
    OR\_E \to AND\_E\ |\ AND\_E\ OR\_OP\ OR\_E,\ OR\_OP \to \textbf{||}\\
    AND\_E \to NOT\_E\ |\ NOT\_E\ AND\_OP\ NOT\_E,\ AND\_OP \to \textbf{\&\&}\\
    NOT\_E = NOT\_OP ?\ CMP\_E,\ NOT\_OP \to !\\
    CMP\_E \to SUM\_E\ |\ SUM\_E\ CMP\_OP\ SUM\_E\\
    CMP\_OP \to \textbf{==}\ |\ \textbf{!=}\ |\ \textbf{<=}\ |\ \textbf{>=}\ |\
    \textbf{<}\ |\ \textbf{>}\\
    SUM\_E \to MUL\_E\ |\ SUM\_E\ SUM\_OP\ MUL\_E,\ SUM\_OP \to \textbf{+}\ |\ \textbf{-}\\
    MUL\_E \to MINUS\_E\ |\ MUL\_E\ MUL\_OP\ MINUS\_E,\ MUL\_OP \to \textbf{*}\ |\ \textbf{/}\\
    MINUS\_E \to -? POW\_E
    POW\_E \to TERM\ |\ TERM\ POW\_OP\ POW\_E,\ POW\_OP \to \textasciicircum\\
    TERM \to NUM\ |\ IDENT\ |\ LBR\ EXPR\ RBR\\
    NUM \to 0 | (1..9) (0..9)^*
\end{gather*}
\end{document}