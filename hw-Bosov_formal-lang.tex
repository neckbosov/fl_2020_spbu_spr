\documentclass[14pt]{extarticle}

\usepackage{cmap}
\usepackage[T2A]{fontenc}
\usepackage[utf8]{inputenc}
\usepackage[russian]{babel}
\usepackage{graphicx}
\usepackage{amsthm,amsmath,amssymb}
\usepackage[russian,colorlinks=true,urlcolor=red,linkcolor=blue]{hyperref}
\usepackage{enumerate}
\usepackage{datetime}
\usepackage{minted}
\usepackage{fancyhdr}
\usepackage{lastpage}
\usepackage{color}
\usepackage{verbatim}
\usepackage{amsthm,amsfonts,amscd}
\usepackage{blkarray}
\usepackage{mathtools}
\usepackage{xcolor,cancel}
\usepackage{tikz}
\usepackage{ifthen}
\usepackage{setspace}
\usepackage{epigraph}
\usepackage{enumitem}

\parskip=0em
\parindent=0em

\sloppy
\voffset=-20mm
\textheight=235mm
\hoffset=-25mm
\textwidth=180mm
\headsep=12pt
\footskip=20pt

\setcounter{page}{0}
\pagestyle{empty}

% Основные математические символы
\newcommand{\N}{\mathbb{N}}   % Natural numbers
\newcommand{\R}{\mathbb{R}}   % Ratio numbers
\newcommand{\Z}{\mathbb{Z}}   % Integer numbers
\def\EPS{\varepsilon}         %
\def\SO{\Rightarrow}          % =>
\def\EQ{\Leftrightarrow}      % <=>
\def\t{\texttt}               %
\def\O{\mathcal{O}}           %
\def\NO{\t{\#}}               % #
\renewcommand{\le}{\leqslant} % <=, beauty
\renewcommand{\ge}{\geqslant} % >=, beauty
\def\XOR{\text{ {\raisebox{-2pt}{\ensuremath{\Hat{}}}} }}
\newcommand{\q}[1]{\langle #1 \rangle}               % <x>
\newcommand\URL[1]{{\footnotesize{\url{#1}}}}        %
\newcommand{\sfrac}[2]{{\scriptstyle\frac{#1}{#2}}}  % Очень маленькая дробь
\newcommand{\mfrac}[2]{{\textstyle\frac{#1}{#2}}}    % Небольшая дробь
\newcommand{\score}[1]{{\bf\color{red}{(#1)}}}

\DeclareMathOperator\arcsinh{arcsinh}
\DeclareMathOperator\arccosh{arccosh}
\DeclareMathOperator\arctanh{arctanh}
% Отступы
\def\makeparindent{\hspace*{\parindent}}
\def\up{\vspace*{-\baselineskip}}
\def\down{\vspace*{\baselineskip}}
\def\LINE{\vspace*{-1em}\noindent \underline{\hbox to 1\textwidth{{ } \hfil{ } \hfil{ } }}}

\lhead{Теория формальных языков, весна 2019/20}
\chead{}
\rhead{Никита Босов}
\renewcommand{\headrulewidth}{0.4pt}

\lfoot{}
\cfoot{\thepage\t{/}\pageref*{LastPage}}
\rfoot{}
\renewcommand{\footrulewidth}{0.4pt}

\newenvironment{MyList}[1][4pt]{
  \begin{enumerate}[1.]
  \setlength{\parskip}{0pt}
  \setlength{\itemsep}{#1}
}{       
  \end{enumerate}
}
\newenvironment{InnerMyList}[1][0pt]{
  \vspace*{-0.5em}
  \begin{enumerate}[a)]
  \setlength{\parskip}{#1}
  \setlength{\itemsep}{0pt}
}{
  \end{enumerate}
}

\newcommand{\Section}[1]{
  \refstepcounter{section}
  \addcontentsline{toc}{section}{\arabic{section}. #1} 
  %{\LARGE \bf \arabic{section}. #1} 
  {\LARGE \bf #1} 
  \vspace*{1em}
  \makeparindent\unskip
}
\newcommand{\Subsection}[1]{
  \refstepcounter{subsection}
  \addcontentsline{toc}{subsection}{\arabic{section}.\arabic{subsection}. #1} 
  {\Large \bf \arabic{section}.\arabic{subsection}. #1} 
  \vspace*{0.5em}
  \makeparindent\unskip
}

% Код с правильными отступами
\newenvironment{code}{
  \VerbatimEnvironment

  \vspace*{-0.5em}
  \begin{minted}{c}{}
  \end{minted}
  \vspace*{-0.5em}

}

% Формулы с правильными отступами
\newenvironment{smallformula}{
 
  \vspace*{-0.8em}
}{
  \vspace*{-1.2em}
  
}
\newenvironment{formula}{
 
  \vspace*{-0.4em}
}{
  \vspace*{-0.6em}
  
}

\definecolor{dkgreen}{rgb}{0,0.6,0}
\definecolor{brown}{rgb}{0.5,0.5,0}
\newcommand{\red}[1]{{\color{red}{#1}}}
\newcommand{\dkgreen}[1]{{\color{dkgreen}{#1}}}
\makeatletter
\AddEnumerateCounter{\asbuk}{\russian@alph}{щ}
\makeatother
\makeatletter
\AddEnumerateCounter{\Asbuk}{\russian@Alph}{Щ}
\makeatother

\begin{document}

\renewcommand{\dateseparator}{--}
\begin{center}
    {\Large\bf
        Второй курс, весенний семестр 2019/20\\
        Домашнее задание \NO5 \\
    }
\end{center}

\vspace{-1em}
\LINE
\vspace{1em}


\pagestyle{fancy}
\begin{enumerate}
    \item Грамматика:
          \begin{gather*}
              SEQ \to (INSTRUCTION\ WS^+)^*\\
              WS \to \textbf{любые пробельные символы и переводы строк}\\
              SPACE \to \textbf{' '},\ LBR \to \textbf{(},\ RBR \to \textbf{)}\\
              INSTRUCTION \to IF | ASSIGN | WHILE | READ | WRITE \\
              BLOCK \to \{ WS^*\ SEQ\ WS^*\}\\
              IF \to \textbf{if}\ SPACE\ LBR\ EXPR\ RBR \ SPACE?\ BLOCK \\
              (WS^+\ \textbf{else } \ SPACE?\ BLOCK)?\\
              ASSIGN \to IDENT\ SPACE^*\ \textbf{=}\ SPACE^*\ EXPR\text{;}\\
              WHILE \to \textbf{while } SPACE\ LBR\ EXPR\ RBR \ SPACE?\ BLOCK\\
              READ \to \textbf{read } LBR\ IDENT \ RBR \text{;}\\
              WRITE \to \textbf{write } LBR\ EXPR \ RBR \text{;}\\
              EXPR \to OR\_E\\
              OR\_E \to AND\_E\ |\ AND\_E\ WS^*\ OR\_OP\ WS^*\ OR\_E\\
              OR\_OP \to \textbf{||}\\
              AND\_E \to CMP\_E\ |\ CMP\_E\ WS^*\ AND\_OP\ WS^*\ CMP\_E\\
              AND\_OP \to \textbf{\&\&}\\
              CMP\_E \to ARITH\_E\ |\ ARITH\_E\ WS^*\ CMP\_OP\ WS^*\ ARITH\_E\\
              CMP\_OP \to \textbf{==}\ |\ \textbf{!=}\ |\ \textbf{<=}\ |\ \textbf{>=}\ |\
              \textbf{<}\ |\ \textbf{>}\\
              ARITH\_E \to SUM\_E\\
              SUM\_E \to MUL\_E\ |\ SUM\_E\ WS^*\ SUM\_OP\ WS^*\ MUL\_E\\
              SUM\_OP \to \textbf{+}\ |\ \textbf{-}\\
              MUL\_E \to POW\_E\ |\ MUL\_E\ WS^*\ MUL\_OP\ WS^*\ POW\_E\\
              MUL\_OP \to \textbf{*}\ |\ \textbf{/}\\
              POW\_E \to TERM\ |\ TERM\ WS^*\ POW\_OP\ WS^*\ POW\_E\\
              POW\_OP \to \textasciicircum\\
              TERM \to NUM\ |\ IDENT\ |\ -?\ LBR\ EXPR\ RBR\\
              NUM \to -?(0 | (1..9) (0..9)^*)\\
              IDENT \to ((a..z) | (A..Z) | \_ ) ((a..z) | (A..Z) | \_ | (0..9)) ^ *
          \end{gather*}

          Более формально: программа --- последовательность инструкций, разделенных точкой с
          запятой. Так же есть блоки кода, их разделять точкой с запятой не нужно. Поддерживаются
          инструкции условного перехода, цикл while, чтение переменных с помощью read и запись с помощью
          write. Все эти конструкции выглядят так же, как в $C$-подобных языках. Также поддерживаются
          арифметические и логические выражения с естественными ассоциативностью и приоритетом.
    \item Действуем последовательно:
          \begin{gather*}
              S \to R\ S\ |\ R\\
              R \to a\ S\ b\ |\ c\ R\ d\ |\ a\ b\ |\ c\ d\ |\ \epsilon\\
              \\
              S \to R\ S\ |\ R\\
              R \to a\ SB\ |\ c\ RD\ |\ a\ b\ |\ c\ d\ |\ \epsilon\\
              SB \to S\ b\\
              RD \to R\ d\\
              \\
              S \to R\ S\ |\ R\ |\ \epsilon\\
              R \to a\ SB\ |\ c\ RD\ |\ a\ b\ |\ c\ d\ \\
              SB \to S\ b\\
              RD \to R\ d\\
              \\
              S \to R\ S\ |\ a\ SB\ |\ c\ RD\ |\ a\ b\ |\ c\ d\ |\ \epsilon\\
              R \to a\ SB\ |\ c\ RD\ |\ a\ b\ |\ c\ d\ \\
              SB \to S\ b\\
              RD \to R\ d\\
          \end{gather*}
          \begin{gather*}
              S \to R\ S\ |\ A\ SB\ |\ C\ RD\ |\ A\ B\ |\ C\ D\ |\ \epsilon\\
              R \to A\ SB\ |\ C\ RD\ |\ A\ B\ |\ C\ D\ \\
              SB \to S\ B\\
              RD \to R\ D\\
              A \to a\\
              B \to b\\
              C \to c\\
              D \to d
          \end{gather*}
    \item Да, является. Грамматика:

          \begin{gather*}
              S \to aa | ab | bb\\
              S \to a S b\\
              S \to aaS\\
              S \to Sbb
          \end{gather*}
\end{enumerate}
\end{document}